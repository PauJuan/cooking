\documentclass[%
a4paper,
hidelinks,
% twoside,
11pt
]{article}

% encoding, font, language
\usepackage[T1]{fontenc}
\usepackage[latin1]{inputenc}
\usepackage{lmodern}
\usepackage{nicefrac}

% Load cookbook package
\usepackage[
%    handwritten,  % font is not available
    nowarnings,
    %myconfig
]
{xcookybooky}

% Make command for degree symbol
\DeclareRobustCommand{\textcelcius}{\ensuremath{^{\circ}\mathrm{C}}}

% Fix index in PDFs
\setcounter{secnumdepth}{1}
\renewcommand*{\recipesection}[2][]
{%
    \subsection[#1]{#2}
}
\renewcommand{\subsectionmark}[1]
{% no implementation to display the section name instead
}

\usepackage{hyperref}    % must be the last package
\hypersetup{%
    pdfauthor            = {Pau Juan-Garcia},
    pdftitle             = {Personal cooking book},
    pdfsubject           = {Recipes},
    pdfkeywords          = {cooking, recipes, cookbook},
    pdfstartview         = {FitV},
    pdfview              = {FitH},
    pdfpagemode          = {UseNone}, % Options; UseNone, UseOutlines
    bookmarksopen        = {true},
    pdfpagetransition    = {Glitter},
    colorlinks           = {true},
    linkcolor            = {black},
    urlcolor             = {blue},
    citecolor            = {black},
    filecolor            = {black},
}

\hbadness=10000	% Ignore underfull boxes

\begin{document}

\title{Pau's recipes cookbook}
\author{Pau Juan-Garcia\\ \href{mailto:Pambientoleg@gmail.com}{Pambientoleg@gmail.com}}
\maketitle

\begin{abstract}
    \noindent This is a cookbook with a mixture of recipes that I have picked up through the years. They include vegan food and use ingredients from a range of nationalities.
\end{abstract}

\tableofcontents

\vspace{5em}

\section{Recipes}
A selection of recipes to have a nice and balanced diet through the year.

% background graphic
\setBackgroundPicture
[
	x, 
	y=-2cm,
	width=\paperwidth-4cm,
	height,
	orientation = pagecenter
]%
{pics/background}

% input recipes
% Complete recipe example
\begin{recipe}
  [% indications (optional)
    preparationtime = {\unit[1]{h}},
    bakingtime={\unit[1]{h}},
    bakingtemperature={\protect\bakingtemperature{
        fanoven=\unit[230]{\textcelcius},
        topbottomheat=\unit[195]{\textcelcius},
        topheat=\unit[195]{\textcelcius},
        gasstove=Level 2}},
    portion = {\portion{5-6}},
    calory={\unit[3]{kJ}},
    source = {Somebody}
]
% title
{Test Recipe}
    
    \graph
    {%
        small=pics/glass,     % small picture
        big=pics/ingredients  % big picture
    }
    
    \introduction{%
        This is an amazing recipe
    }
    
    \ingredients)\\
        3 & Eggs\\
        \unit[200]{ml} & Cream\\
        40 g & Sugar\\
        50 g & Butter
    }
    
    \preparation{%
        \step Mix all the ingredients
        \step Put them in the oven
        \step Eat it
    }
    
    \suggestion[Suggestion]
    {%
        Add more chocolate
    }
    
    \hint{%
        Eat in hot
    }
    
\end{recipe}

% Complete recipe example
\begin{recipe}
  [% indications (optional)
    preparationtime = {\unit[1]{h}},
    bakingtime={\unit[1]{h}},
    bakingtemperature={\protect\bakingtemperature{
        fanoven=\unit[230]{\textcelcius},
        topbottomheat=\unit[195]{\textcelcius},
        topheat=\unit[195]{\textcelcius},
        gasstove=Level 2}},
    portion = {\portion{5-6}},
    calory={\unit[3]{kJ}},
    source = {Somebody}
]
% title
{Test Recipe 2}
    
    \graph
    {%
        small=pics/glass,     % small picture
        big=pics/ingredients  % big picture
    }
    
    \introduction{%
        This is an amazing recipe
    }
    
    \ingredients)\\
        3 & Eggs\\
        \unit[200]{ml} & Cream\\
        40 g & Sugar\\
        50 g & Butter
    }
    
    \preparation{%
        \step Mix all the ingredients
        \step Put them in the oven
        \step Eat it
    }
    
    \suggestion[Suggestion]
    {%
        Add more chocolate
    }
    
    \hint{%
        Eat in hot
    }
    
\end{recipe}

% Complete recipe example
\begin{recipe}
  [% indications (optional)
    preparationtime = {\unit[1]{h}},
    bakingtime={\unit[1]{h}},
    bakingtemperature={\protect\bakingtemperature{
        fanoven=\unit[230]{\textcelcius},
        topbottomheat=\unit[195]{\textcelcius},
        topheat=\unit[195]{\textcelcius},
        gasstove=Level 2}},
    portion = {\portion{5-6}},
    calory={\unit[3]{kJ}},
    source = {Somebody}
]
% title
{Test Recipe 3}
    
    \graph
    {%
        small=pics/glass,     % small picture
        big=pics/ingredients  % big picture
    }
    
    \introduction{%
        This is an amazing recipe
    }
    
    \ingredients)\\
        3 & Eggs\\
        \unit[200]{ml} & Cream\\
        40 g & Sugar\\
        50 g & Butter
    }
    
    \preparation{%
        \step Mix all the ingredients
        \step Put them in the oven
        \step Eat it
    }
    
    \suggestion[Suggestion]
    {%
        Add more chocolate
    }
    
    \hint{%
        Eat in hot
    }
    
\end{recipe}

%% Complete recipe example
\begin{recipe}
  [% indications (optional)
    preparationtime = {\unit[1]{h}},
    bakingtime={\unit[1]{h}},
    bakingtemperature={\protect\bakingtemperature{
        fanoven=\unit[230]{\textcelcius},
        topbottomheat=\unit[195]{\textcelcius},
        topheat=\unit[195]{\textcelcius},
        gasstove=Level 2}},
    portion = {\portion{5-6}},
    calory={\unit[3]{kJ}},
    source = {Somebody}
]
% title
{Test Recipe}
    
    \graph
    {%
        small=pics/glass,     % small picture
        big=pics/ingredients  % big picture
    }
    
    \introduction{%
        This is an amazing recipe
    }
    
    \ingredients)\\
        3 & Eggs\\
        \unit[200]{ml} & Cream\\
        40 g & Sugar\\
        50 g & Butter
    }
    
    \preparation{%
        \step Mix all the ingredients
        \step Put them in the oven
        \step Eat it
    }
    
    \suggestion[Suggestion]
    {%
        Add more chocolate
    }
    
    \hint{%
        Eat in hot
    }
    
\end{recipe}

%% Complete recipe example
\begin{recipe}
  [% indications (optional)
    preparationtime = {\unit[1]{h}},
    bakingtime={\unit[1]{h}},
    bakingtemperature={\protect\bakingtemperature{
        fanoven=\unit[230]{\textcelcius},
        topbottomheat=\unit[195]{\textcelcius},
        topheat=\unit[195]{\textcelcius},
        gasstove=Level 2}},
    portion = {\portion{5-6}},
    calory={\unit[3]{kJ}},
    source = {Somebody}
]
% title
{Test Recipe}
    
    \graph
    {%
        small=pics/glass,     % small picture
        big=pics/ingredients  % big picture
    }
    
    \introduction{%
        This is an amazing recipe
    }
    
    \ingredients)\\
        3 & Eggs\\
        \unit[200]{ml} & Cream\\
        40 g & Sugar\\
        50 g & Butter
    }
    
    \preparation{%
        \step Mix all the ingredients
        \step Put them in the oven
        \step Eat it
    }
    
    \suggestion[Suggestion]
    {%
        Add more chocolate
    }
    
    \hint{%
        Eat in hot
    }
    
\end{recipe}

%% Complete recipe example
\begin{recipe}
  [% indications (optional)
    preparationtime = {\unit[1]{h}},
    bakingtime={\unit[1]{h}},
    bakingtemperature={\protect\bakingtemperature{
        fanoven=\unit[230]{\textcelcius},
        topbottomheat=\unit[195]{\textcelcius},
        topheat=\unit[195]{\textcelcius},
        gasstove=Level 2}},
    portion = {\portion{5-6}},
    calory={\unit[3]{kJ}},
    source = {Somebody}
]
% title
{Test Recipe}
    
    \graph
    {%
        small=pics/glass,     % small picture
        big=pics/ingredients  % big picture
    }
    
    \introduction{%
        This is an amazing recipe
    }
    
    \ingredients)\\
        3 & Eggs\\
        \unit[200]{ml} & Cream\\
        40 g & Sugar\\
        50 g & Butter
    }
    
    \preparation{%
        \step Mix all the ingredients
        \step Put them in the oven
        \step Eat it
    }
    
    \suggestion[Suggestion]
    {%
        Add more chocolate
    }
    
    \hint{%
        Eat in hot
    }
    
\end{recipe}

%% Complete recipe example
\begin{recipe}
  [% indications (optional)
    preparationtime = {\unit[1]{h}},
    bakingtime={\unit[1]{h}},
    bakingtemperature={\protect\bakingtemperature{
        fanoven=\unit[230]{\textcelcius},
        topbottomheat=\unit[195]{\textcelcius},
        topheat=\unit[195]{\textcelcius},
        gasstove=Level 2}},
    portion = {\portion{5-6}},
    calory={\unit[3]{kJ}},
    source = {Somebody}
]
% title
{Test Recipe}
    
    \graph
    {%
        small=pics/glass,     % small picture
        big=pics/ingredients  % big picture
    }
    
    \introduction{%
        This is an amazing recipe
    }
    
    \ingredients)\\
        3 & Eggs\\
        \unit[200]{ml} & Cream\\
        40 g & Sugar\\
        50 g & Butter
    }
    
    \preparation{%
        \step Mix all the ingredients
        \step Put them in the oven
        \step Eat it
    }
    
    \suggestion[Suggestion]
    {%
        Add more chocolate
    }
    
    \hint{%
        Eat in hot
    }
    
\end{recipe}

%% Complete recipe example
\begin{recipe}
  [% indications (optional)
    preparationtime = {\unit[1]{h}},
    bakingtime={\unit[1]{h}},
    bakingtemperature={\protect\bakingtemperature{
        fanoven=\unit[230]{\textcelcius},
        topbottomheat=\unit[195]{\textcelcius},
        topheat=\unit[195]{\textcelcius},
        gasstove=Level 2}},
    portion = {\portion{5-6}},
    calory={\unit[3]{kJ}},
    source = {Somebody}
]
% title
{Test Recipe}
    
    \graph
    {%
        small=pics/glass,     % small picture
        big=pics/ingredients  % big picture
    }
    
    \introduction{%
        This is an amazing recipe
    }
    
    \ingredients)\\
        3 & Eggs\\
        \unit[200]{ml} & Cream\\
        40 g & Sugar\\
        50 g & Butter
    }
    
    \preparation{%
        \step Mix all the ingredients
        \step Put them in the oven
        \step Eat it
    }
    
    \suggestion[Suggestion]
    {%
        Add more chocolate
    }
    
    \hint{%
        Eat in hot
    }
    
\end{recipe}

%% Complete recipe example
\begin{recipe}
  [% indications (optional)
    preparationtime = {\unit[1]{h}},
    bakingtime={\unit[1]{h}},
    bakingtemperature={\protect\bakingtemperature{
        fanoven=\unit[230]{\textcelcius},
        topbottomheat=\unit[195]{\textcelcius},
        topheat=\unit[195]{\textcelcius},
        gasstove=Level 2}},
    portion = {\portion{5-6}},
    calory={\unit[3]{kJ}},
    source = {Somebody}
]
% title
{Test Recipe}
    
    \graph
    {%
        small=pics/glass,     % small picture
        big=pics/ingredients  % big picture
    }
    
    \introduction{%
        This is an amazing recipe
    }
    
    \ingredients)\\
        3 & Eggs\\
        \unit[200]{ml} & Cream\\
        40 g & Sugar\\
        50 g & Butter
    }
    
    \preparation{%
        \step Mix all the ingredients
        \step Put them in the oven
        \step Eat it
    }
    
    \suggestion[Suggestion]
    {%
        Add more chocolate
    }
    
    \hint{%
        Eat in hot
    }
    
\end{recipe}

%% Complete recipe example
\begin{recipe}
  [% indications (optional)
    preparationtime = {\unit[1]{h}},
    bakingtime={\unit[1]{h}},
    bakingtemperature={\protect\bakingtemperature{
        fanoven=\unit[230]{\textcelcius},
        topbottomheat=\unit[195]{\textcelcius},
        topheat=\unit[195]{\textcelcius},
        gasstove=Level 2}},
    portion = {\portion{5-6}},
    calory={\unit[3]{kJ}},
    source = {Somebody}
]
% title
{Test Recipe}
    
    \graph
    {%
        small=pics/glass,     % small picture
        big=pics/ingredients  % big picture
    }
    
    \introduction{%
        This is an amazing recipe
    }
    
    \ingredients)\\
        3 & Eggs\\
        \unit[200]{ml} & Cream\\
        40 g & Sugar\\
        50 g & Butter
    }
    
    \preparation{%
        \step Mix all the ingredients
        \step Put them in the oven
        \step Eat it
    }
    
    \suggestion[Suggestion]
    {%
        Add more chocolate
    }
    
    \hint{%
        Eat in hot
    }
    
\end{recipe}

%% Complete recipe example
\begin{recipe}
  [% indications (optional)
    preparationtime = {\unit[1]{h}},
    bakingtime={\unit[1]{h}},
    bakingtemperature={\protect\bakingtemperature{
        fanoven=\unit[230]{\textcelcius},
        topbottomheat=\unit[195]{\textcelcius},
        topheat=\unit[195]{\textcelcius},
        gasstove=Level 2}},
    portion = {\portion{5-6}},
    calory={\unit[3]{kJ}},
    source = {Somebody}
]
% title
{Test Recipe}
    
    \graph
    {%
        small=pics/glass,     % small picture
        big=pics/ingredients  % big picture
    }
    
    \introduction{%
        This is an amazing recipe
    }
    
    \ingredients)\\
        3 & Eggs\\
        \unit[200]{ml} & Cream\\
        40 g & Sugar\\
        50 g & Butter
    }
    
    \preparation{%
        \step Mix all the ingredients
        \step Put them in the oven
        \step Eat it
    }
    
    \suggestion[Suggestion]
    {%
        Add more chocolate
    }
    
    \hint{%
        Eat in hot
    }
    
\end{recipe}

%% Complete recipe example
\begin{recipe}
  [% indications (optional)
    preparationtime = {\unit[1]{h}},
    bakingtime={\unit[1]{h}},
    bakingtemperature={\protect\bakingtemperature{
        fanoven=\unit[230]{\textcelcius},
        topbottomheat=\unit[195]{\textcelcius},
        topheat=\unit[195]{\textcelcius},
        gasstove=Level 2}},
    portion = {\portion{5-6}},
    calory={\unit[3]{kJ}},
    source = {Somebody}
]
% title
{Test Recipe}
    
    \graph
    {%
        small=pics/glass,     % small picture
        big=pics/ingredients  % big picture
    }
    
    \introduction{%
        This is an amazing recipe
    }
    
    \ingredients)\\
        3 & Eggs\\
        \unit[200]{ml} & Cream\\
        40 g & Sugar\\
        50 g & Butter
    }
    
    \preparation{%
        \step Mix all the ingredients
        \step Put them in the oven
        \step Eat it
    }
    
    \suggestion[Suggestion]
    {%
        Add more chocolate
    }
    
    \hint{%
        Eat in hot
    }
    
\end{recipe}

%% Complete recipe example
\begin{recipe}
  [% indications (optional)
    preparationtime = {\unit[1]{h}},
    bakingtime={\unit[1]{h}},
    bakingtemperature={\protect\bakingtemperature{
        fanoven=\unit[230]{\textcelcius},
        topbottomheat=\unit[195]{\textcelcius},
        topheat=\unit[195]{\textcelcius},
        gasstove=Level 2}},
    portion = {\portion{5-6}},
    calory={\unit[3]{kJ}},
    source = {Somebody}
]
% title
{Test Recipe}
    
    \graph
    {%
        small=pics/glass,     % small picture
        big=pics/ingredients  % big picture
    }
    
    \introduction{%
        This is an amazing recipe
    }
    
    \ingredients)\\
        3 & Eggs\\
        \unit[200]{ml} & Cream\\
        40 g & Sugar\\
        50 g & Butter
    }
    
    \preparation{%
        \step Mix all the ingredients
        \step Put them in the oven
        \step Eat it
    }
    
    \suggestion[Suggestion]
    {%
        Add more chocolate
    }
    
    \hint{%
        Eat in hot
    }
    
\end{recipe}

%% Complete recipe example
\begin{recipe}
  [% indications (optional)
    preparationtime = {\unit[1]{h}},
    bakingtime={\unit[1]{h}},
    bakingtemperature={\protect\bakingtemperature{
        fanoven=\unit[230]{\textcelcius},
        topbottomheat=\unit[195]{\textcelcius},
        topheat=\unit[195]{\textcelcius},
        gasstove=Level 2}},
    portion = {\portion{5-6}},
    calory={\unit[3]{kJ}},
    source = {Somebody}
]
% title
{Test Recipe}
    
    \graph
    {%
        small=pics/glass,     % small picture
        big=pics/ingredients  % big picture
    }
    
    \introduction{%
        This is an amazing recipe
    }
    
    \ingredients)\\
        3 & Eggs\\
        \unit[200]{ml} & Cream\\
        40 g & Sugar\\
        50 g & Butter
    }
    
    \preparation{%
        \step Mix all the ingredients
        \step Put them in the oven
        \step Eat it
    }
    
    \suggestion[Suggestion]
    {%
        Add more chocolate
    }
    
    \hint{%
        Eat in hot
    }
    
\end{recipe}

%% Complete recipe example
\begin{recipe}
  [% indications (optional)
    preparationtime = {\unit[1]{h}},
    bakingtime={\unit[1]{h}},
    bakingtemperature={\protect\bakingtemperature{
        fanoven=\unit[230]{\textcelcius},
        topbottomheat=\unit[195]{\textcelcius},
        topheat=\unit[195]{\textcelcius},
        gasstove=Level 2}},
    portion = {\portion{5-6}},
    calory={\unit[3]{kJ}},
    source = {Somebody}
]
% title
{Test Recipe}
    
    \graph
    {%
        small=pics/glass,     % small picture
        big=pics/ingredients  % big picture
    }
    
    \introduction{%
        This is an amazing recipe
    }
    
    \ingredients)\\
        3 & Eggs\\
        \unit[200]{ml} & Cream\\
        40 g & Sugar\\
        50 g & Butter
    }
    
    \preparation{%
        \step Mix all the ingredients
        \step Put them in the oven
        \step Eat it
    }
    
    \suggestion[Suggestion]
    {%
        Add more chocolate
    }
    
    \hint{%
        Eat in hot
    }
    
\end{recipe}

%% Complete recipe example
\begin{recipe}
  [% indications (optional)
    preparationtime = {\unit[1]{h}},
    bakingtime={\unit[1]{h}},
    bakingtemperature={\protect\bakingtemperature{
        fanoven=\unit[230]{\textcelcius},
        topbottomheat=\unit[195]{\textcelcius},
        topheat=\unit[195]{\textcelcius},
        gasstove=Level 2}},
    portion = {\portion{5-6}},
    calory={\unit[3]{kJ}},
    source = {Somebody}
]
% title
{Test Recipe}
    
    \graph
    {%
        small=pics/glass,     % small picture
        big=pics/ingredients  % big picture
    }
    
    \introduction{%
        This is an amazing recipe
    }
    
    \ingredients)\\
        3 & Eggs\\
        \unit[200]{ml} & Cream\\
        40 g & Sugar\\
        50 g & Butter
    }
    
    \preparation{%
        \step Mix all the ingredients
        \step Put them in the oven
        \step Eat it
    }
    
    \suggestion[Suggestion]
    {%
        Add more chocolate
    }
    
    \hint{%
        Eat in hot
    }
    
\end{recipe}

%% Complete recipe example
\begin{recipe}
  [% indications (optional)
    preparationtime = {\unit[1]{h}},
    bakingtime={\unit[1]{h}},
    bakingtemperature={\protect\bakingtemperature{
        fanoven=\unit[230]{\textcelcius},
        topbottomheat=\unit[195]{\textcelcius},
        topheat=\unit[195]{\textcelcius},
        gasstove=Level 2}},
    portion = {\portion{5-6}},
    calory={\unit[3]{kJ}},
    source = {Somebody}
]
% title
{Test Recipe}
    
    \graph
    {%
        small=pics/glass,     % small picture
        big=pics/ingredients  % big picture
    }
    
    \introduction{%
        This is an amazing recipe
    }
    
    \ingredients)\\
        3 & Eggs\\
        \unit[200]{ml} & Cream\\
        40 g & Sugar\\
        50 g & Butter
    }
    
    \preparation{%
        \step Mix all the ingredients
        \step Put them in the oven
        \step Eat it
    }
    
    \suggestion[Suggestion]
    {%
        Add more chocolate
    }
    
    \hint{%
        Eat in hot
    }
    
\end{recipe}

%% Complete recipe example
\begin{recipe}
  [% indications (optional)
    preparationtime = {\unit[1]{h}},
    bakingtime={\unit[1]{h}},
    bakingtemperature={\protect\bakingtemperature{
        fanoven=\unit[230]{\textcelcius},
        topbottomheat=\unit[195]{\textcelcius},
        topheat=\unit[195]{\textcelcius},
        gasstove=Level 2}},
    portion = {\portion{5-6}},
    calory={\unit[3]{kJ}},
    source = {Somebody}
]
% title
{Test Recipe}
    
    \graph
    {%
        small=pics/glass,     % small picture
        big=pics/ingredients  % big picture
    }
    
    \introduction{%
        This is an amazing recipe
    }
    
    \ingredients)\\
        3 & Eggs\\
        \unit[200]{ml} & Cream\\
        40 g & Sugar\\
        50 g & Butter
    }
    
    \preparation{%
        \step Mix all the ingredients
        \step Put them in the oven
        \step Eat it
    }
    
    \suggestion[Suggestion]
    {%
        Add more chocolate
    }
    
    \hint{%
        Eat in hot
    }
    
\end{recipe}

%% Complete recipe example
\begin{recipe}
  [% indications (optional)
    preparationtime = {\unit[1]{h}},
    bakingtime={\unit[1]{h}},
    bakingtemperature={\protect\bakingtemperature{
        fanoven=\unit[230]{\textcelcius},
        topbottomheat=\unit[195]{\textcelcius},
        topheat=\unit[195]{\textcelcius},
        gasstove=Level 2}},
    portion = {\portion{5-6}},
    calory={\unit[3]{kJ}},
    source = {Somebody}
]
% title
{Test Recipe}
    
    \graph
    {%
        small=pics/glass,     % small picture
        big=pics/ingredients  % big picture
    }
    
    \introduction{%
        This is an amazing recipe
    }
    
    \ingredients)\\
        3 & Eggs\\
        \unit[200]{ml} & Cream\\
        40 g & Sugar\\
        50 g & Butter
    }
    
    \preparation{%
        \step Mix all the ingredients
        \step Put them in the oven
        \step Eat it
    }
    
    \suggestion[Suggestion]
    {%
        Add more chocolate
    }
    
    \hint{%
        Eat in hot
    }
    
\end{recipe}

%% Complete recipe example
\begin{recipe}
  [% indications (optional)
    preparationtime = {\unit[1]{h}},
    bakingtime={\unit[1]{h}},
    bakingtemperature={\protect\bakingtemperature{
        fanoven=\unit[230]{\textcelcius},
        topbottomheat=\unit[195]{\textcelcius},
        topheat=\unit[195]{\textcelcius},
        gasstove=Level 2}},
    portion = {\portion{5-6}},
    calory={\unit[3]{kJ}},
    source = {Somebody}
]
% title
{Test Recipe}
    
    \graph
    {%
        small=pics/glass,     % small picture
        big=pics/ingredients  % big picture
    }
    
    \introduction{%
        This is an amazing recipe
    }
    
    \ingredients)\\
        3 & Eggs\\
        \unit[200]{ml} & Cream\\
        40 g & Sugar\\
        50 g & Butter
    }
    
    \preparation{%
        \step Mix all the ingredients
        \step Put them in the oven
        \step Eat it
    }
    
    \suggestion[Suggestion]
    {%
        Add more chocolate
    }
    
    \hint{%
        Eat in hot
    }
    
\end{recipe}

%% Complete recipe example
\begin{recipe}
  [% indications (optional)
    preparationtime = {\unit[1]{h}},
    bakingtime={\unit[1]{h}},
    bakingtemperature={\protect\bakingtemperature{
        fanoven=\unit[230]{\textcelcius},
        topbottomheat=\unit[195]{\textcelcius},
        topheat=\unit[195]{\textcelcius},
        gasstove=Level 2}},
    portion = {\portion{5-6}},
    calory={\unit[3]{kJ}},
    source = {Somebody}
]
% title
{Test Recipe}
    
    \graph
    {%
        small=pics/glass,     % small picture
        big=pics/ingredients  % big picture
    }
    
    \introduction{%
        This is an amazing recipe
    }
    
    \ingredients)\\
        3 & Eggs\\
        \unit[200]{ml} & Cream\\
        40 g & Sugar\\
        50 g & Butter
    }
    
    \preparation{%
        \step Mix all the ingredients
        \step Put them in the oven
        \step Eat it
    }
    
    \suggestion[Suggestion]
    {%
        Add more chocolate
    }
    
    \hint{%
        Eat in hot
    }
    
\end{recipe}

%% Complete recipe example
\begin{recipe}
  [% indications (optional)
    preparationtime = {\unit[1]{h}},
    bakingtime={\unit[1]{h}},
    bakingtemperature={\protect\bakingtemperature{
        fanoven=\unit[230]{\textcelcius},
        topbottomheat=\unit[195]{\textcelcius},
        topheat=\unit[195]{\textcelcius},
        gasstove=Level 2}},
    portion = {\portion{5-6}},
    calory={\unit[3]{kJ}},
    source = {Somebody}
]
% title
{Test Recipe}
    
    \graph
    {%
        small=pics/glass,     % small picture
        big=pics/ingredients  % big picture
    }
    
    \introduction{%
        This is an amazing recipe
    }
    
    \ingredients)\\
        3 & Eggs\\
        \unit[200]{ml} & Cream\\
        40 g & Sugar\\
        50 g & Butter
    }
    
    \preparation{%
        \step Mix all the ingredients
        \step Put them in the oven
        \step Eat it
    }
    
    \suggestion[Suggestion]
    {%
        Add more chocolate
    }
    
    \hint{%
        Eat in hot
    }
    
\end{recipe}

%% Complete recipe example
\begin{recipe}
  [% indications (optional)
    preparationtime = {\unit[1]{h}},
    bakingtime={\unit[1]{h}},
    bakingtemperature={\protect\bakingtemperature{
        fanoven=\unit[230]{\textcelcius},
        topbottomheat=\unit[195]{\textcelcius},
        topheat=\unit[195]{\textcelcius},
        gasstove=Level 2}},
    portion = {\portion{5-6}},
    calory={\unit[3]{kJ}},
    source = {Somebody}
]
% title
{Test Recipe}
    
    \graph
    {%
        small=pics/glass,     % small picture
        big=pics/ingredients  % big picture
    }
    
    \introduction{%
        This is an amazing recipe
    }
    
    \ingredients)\\
        3 & Eggs\\
        \unit[200]{ml} & Cream\\
        40 g & Sugar\\
        50 g & Butter
    }
    
    \preparation{%
        \step Mix all the ingredients
        \step Put them in the oven
        \step Eat it
    }
    
    \suggestion[Suggestion]
    {%
        Add more chocolate
    }
    
    \hint{%
        Eat in hot
    }
    
\end{recipe}

%% Complete recipe example
\begin{recipe}
  [% indications (optional)
    preparationtime = {\unit[1]{h}},
    bakingtime={\unit[1]{h}},
    bakingtemperature={\protect\bakingtemperature{
        fanoven=\unit[230]{\textcelcius},
        topbottomheat=\unit[195]{\textcelcius},
        topheat=\unit[195]{\textcelcius},
        gasstove=Level 2}},
    portion = {\portion{5-6}},
    calory={\unit[3]{kJ}},
    source = {Somebody}
]
% title
{Test Recipe}
    
    \graph
    {%
        small=pics/glass,     % small picture
        big=pics/ingredients  % big picture
    }
    
    \introduction{%
        This is an amazing recipe
    }
    
    \ingredients)\\
        3 & Eggs\\
        \unit[200]{ml} & Cream\\
        40 g & Sugar\\
        50 g & Butter
    }
    
    \preparation{%
        \step Mix all the ingredients
        \step Put them in the oven
        \step Eat it
    }
    
    \suggestion[Suggestion]
    {%
        Add more chocolate
    }
    
    \hint{%
        Eat in hot
    }
    
\end{recipe}

%% Complete recipe example
\begin{recipe}
  [% indications (optional)
    preparationtime = {\unit[1]{h}},
    bakingtime={\unit[1]{h}},
    bakingtemperature={\protect\bakingtemperature{
        fanoven=\unit[230]{\textcelcius},
        topbottomheat=\unit[195]{\textcelcius},
        topheat=\unit[195]{\textcelcius},
        gasstove=Level 2}},
    portion = {\portion{5-6}},
    calory={\unit[3]{kJ}},
    source = {Somebody}
]
% title
{Test Recipe}
    
    \graph
    {%
        small=pics/glass,     % small picture
        big=pics/ingredients  % big picture
    }
    
    \introduction{%
        This is an amazing recipe
    }
    
    \ingredients)\\
        3 & Eggs\\
        \unit[200]{ml} & Cream\\
        40 g & Sugar\\
        50 g & Butter
    }
    
    \preparation{%
        \step Mix all the ingredients
        \step Put them in the oven
        \step Eat it
    }
    
    \suggestion[Suggestion]
    {%
        Add more chocolate
    }
    
    \hint{%
        Eat in hot
    }
    
\end{recipe}

%% Complete recipe example
\begin{recipe}
  [% indications (optional)
    preparationtime = {\unit[1]{h}},
    bakingtime={\unit[1]{h}},
    bakingtemperature={\protect\bakingtemperature{
        fanoven=\unit[230]{\textcelcius},
        topbottomheat=\unit[195]{\textcelcius},
        topheat=\unit[195]{\textcelcius},
        gasstove=Level 2}},
    portion = {\portion{5-6}},
    calory={\unit[3]{kJ}},
    source = {Somebody}
]
% title
{Test Recipe}
    
    \graph
    {%
        small=pics/glass,     % small picture
        big=pics/ingredients  % big picture
    }
    
    \introduction{%
        This is an amazing recipe
    }
    
    \ingredients)\\
        3 & Eggs\\
        \unit[200]{ml} & Cream\\
        40 g & Sugar\\
        50 g & Butter
    }
    
    \preparation{%
        \step Mix all the ingredients
        \step Put them in the oven
        \step Eat it
    }
    
    \suggestion[Suggestion]
    {%
        Add more chocolate
    }
    
    \hint{%
        Eat in hot
    }
    
\end{recipe}

%% Complete recipe example
\begin{recipe}
  [% indications (optional)
    preparationtime = {\unit[1]{h}},
    bakingtime={\unit[1]{h}},
    bakingtemperature={\protect\bakingtemperature{
        fanoven=\unit[230]{\textcelcius},
        topbottomheat=\unit[195]{\textcelcius},
        topheat=\unit[195]{\textcelcius},
        gasstove=Level 2}},
    portion = {\portion{5-6}},
    calory={\unit[3]{kJ}},
    source = {Somebody}
]
% title
{Test Recipe}
    
    \graph
    {%
        small=pics/glass,     % small picture
        big=pics/ingredients  % big picture
    }
    
    \introduction{%
        This is an amazing recipe
    }
    
    \ingredients)\\
        3 & Eggs\\
        \unit[200]{ml} & Cream\\
        40 g & Sugar\\
        50 g & Butter
    }
    
    \preparation{%
        \step Mix all the ingredients
        \step Put them in the oven
        \step Eat it
    }
    
    \suggestion[Suggestion]
    {%
        Add more chocolate
    }
    
    \hint{%
        Eat in hot
    }
    
\end{recipe}

%% Complete recipe example
\begin{recipe}
  [% indications (optional)
    preparationtime = {\unit[1]{h}},
    bakingtime={\unit[1]{h}},
    bakingtemperature={\protect\bakingtemperature{
        fanoven=\unit[230]{\textcelcius},
        topbottomheat=\unit[195]{\textcelcius},
        topheat=\unit[195]{\textcelcius},
        gasstove=Level 2}},
    portion = {\portion{5-6}},
    calory={\unit[3]{kJ}},
    source = {Somebody}
]
% title
{Test Recipe}
    
    \graph
    {%
        small=pics/glass,     % small picture
        big=pics/ingredients  % big picture
    }
    
    \introduction{%
        This is an amazing recipe
    }
    
    \ingredients)\\
        3 & Eggs\\
        \unit[200]{ml} & Cream\\
        40 g & Sugar\\
        50 g & Butter
    }
    
    \preparation{%
        \step Mix all the ingredients
        \step Put them in the oven
        \step Eat it
    }
    
    \suggestion[Suggestion]
    {%
        Add more chocolate
    }
    
    \hint{%
        Eat in hot
    }
    
\end{recipe}

%% Complete recipe example
\begin{recipe}
  [% indications (optional)
    preparationtime = {\unit[1]{h}},
    bakingtime={\unit[1]{h}},
    bakingtemperature={\protect\bakingtemperature{
        fanoven=\unit[230]{\textcelcius},
        topbottomheat=\unit[195]{\textcelcius},
        topheat=\unit[195]{\textcelcius},
        gasstove=Level 2}},
    portion = {\portion{5-6}},
    calory={\unit[3]{kJ}},
    source = {Somebody}
]
% title
{Test Recipe}
    
    \graph
    {%
        small=pics/glass,     % small picture
        big=pics/ingredients  % big picture
    }
    
    \introduction{%
        This is an amazing recipe
    }
    
    \ingredients)\\
        3 & Eggs\\
        \unit[200]{ml} & Cream\\
        40 g & Sugar\\
        50 g & Butter
    }
    
    \preparation{%
        \step Mix all the ingredients
        \step Put them in the oven
        \step Eat it
    }
    
    \suggestion[Suggestion]
    {%
        Add more chocolate
    }
    
    \hint{%
        Eat in hot
    }
    
\end{recipe}

%% Complete recipe example
\begin{recipe}
  [% indications (optional)
    preparationtime = {\unit[1]{h}},
    bakingtime={\unit[1]{h}},
    bakingtemperature={\protect\bakingtemperature{
        fanoven=\unit[230]{\textcelcius},
        topbottomheat=\unit[195]{\textcelcius},
        topheat=\unit[195]{\textcelcius},
        gasstove=Level 2}},
    portion = {\portion{5-6}},
    calory={\unit[3]{kJ}},
    source = {Somebody}
]
% title
{Test Recipe}
    
    \graph
    {%
        small=pics/glass,     % small picture
        big=pics/ingredients  % big picture
    }
    
    \introduction{%
        This is an amazing recipe
    }
    
    \ingredients)\\
        3 & Eggs\\
        \unit[200]{ml} & Cream\\
        40 g & Sugar\\
        50 g & Butter
    }
    
    \preparation{%
        \step Mix all the ingredients
        \step Put them in the oven
        \step Eat it
    }
    
    \suggestion[Suggestion]
    {%
        Add more chocolate
    }
    
    \hint{%
        Eat in hot
    }
    
\end{recipe}

%% Complete recipe example
\begin{recipe}
  [% indications (optional)
    preparationtime = {\unit[1]{h}},
    bakingtime={\unit[1]{h}},
    bakingtemperature={\protect\bakingtemperature{
        fanoven=\unit[230]{\textcelcius},
        topbottomheat=\unit[195]{\textcelcius},
        topheat=\unit[195]{\textcelcius},
        gasstove=Level 2}},
    portion = {\portion{5-6}},
    calory={\unit[3]{kJ}},
    source = {Somebody}
]
% title
{Test Recipe}
    
    \graph
    {%
        small=pics/glass,     % small picture
        big=pics/ingredients  % big picture
    }
    
    \introduction{%
        This is an amazing recipe
    }
    
    \ingredients)\\
        3 & Eggs\\
        \unit[200]{ml} & Cream\\
        40 g & Sugar\\
        50 g & Butter
    }
    
    \preparation{%
        \step Mix all the ingredients
        \step Put them in the oven
        \step Eat it
    }
    
    \suggestion[Suggestion]
    {%
        Add more chocolate
    }
    
    \hint{%
        Eat in hot
    }
    
\end{recipe}

%% Complete recipe example
\begin{recipe}
  [% indications (optional)
    preparationtime = {\unit[1]{h}},
    bakingtime={\unit[1]{h}},
    bakingtemperature={\protect\bakingtemperature{
        fanoven=\unit[230]{\textcelcius},
        topbottomheat=\unit[195]{\textcelcius},
        topheat=\unit[195]{\textcelcius},
        gasstove=Level 2}},
    portion = {\portion{5-6}},
    calory={\unit[3]{kJ}},
    source = {Somebody}
]
% title
{Test Recipe}
    
    \graph
    {%
        small=pics/glass,     % small picture
        big=pics/ingredients  % big picture
    }
    
    \introduction{%
        This is an amazing recipe
    }
    
    \ingredients)\\
        3 & Eggs\\
        \unit[200]{ml} & Cream\\
        40 g & Sugar\\
        50 g & Butter
    }
    
    \preparation{%
        \step Mix all the ingredients
        \step Put them in the oven
        \step Eat it
    }
    
    \suggestion[Suggestion]
    {%
        Add more chocolate
    }
    
    \hint{%
        Eat in hot
    }
    
\end{recipe}

%% Complete recipe example
\begin{recipe}
  [% indications (optional)
    preparationtime = {\unit[1]{h}},
    bakingtime={\unit[1]{h}},
    bakingtemperature={\protect\bakingtemperature{
        fanoven=\unit[230]{\textcelcius},
        topbottomheat=\unit[195]{\textcelcius},
        topheat=\unit[195]{\textcelcius},
        gasstove=Level 2}},
    portion = {\portion{5-6}},
    calory={\unit[3]{kJ}},
    source = {Somebody}
]
% title
{Test Recipe}
    
    \graph
    {%
        small=pics/glass,     % small picture
        big=pics/ingredients  % big picture
    }
    
    \introduction{%
        This is an amazing recipe
    }
    
    \ingredients)\\
        3 & Eggs\\
        \unit[200]{ml} & Cream\\
        40 g & Sugar\\
        50 g & Butter
    }
    
    \preparation{%
        \step Mix all the ingredients
        \step Put them in the oven
        \step Eat it
    }
    
    \suggestion[Suggestion]
    {%
        Add more chocolate
    }
    
    \hint{%
        Eat in hot
    }
    
\end{recipe}

%% Complete recipe example
\begin{recipe}
  [% indications (optional)
    preparationtime = {\unit[1]{h}},
    bakingtime={\unit[1]{h}},
    bakingtemperature={\protect\bakingtemperature{
        fanoven=\unit[230]{\textcelcius},
        topbottomheat=\unit[195]{\textcelcius},
        topheat=\unit[195]{\textcelcius},
        gasstove=Level 2}},
    portion = {\portion{5-6}},
    calory={\unit[3]{kJ}},
    source = {Somebody}
]
% title
{Test Recipe}
    
    \graph
    {%
        small=pics/glass,     % small picture
        big=pics/ingredients  % big picture
    }
    
    \introduction{%
        This is an amazing recipe
    }
    
    \ingredients)\\
        3 & Eggs\\
        \unit[200]{ml} & Cream\\
        40 g & Sugar\\
        50 g & Butter
    }
    
    \preparation{%
        \step Mix all the ingredients
        \step Put them in the oven
        \step Eat it
    }
    
    \suggestion[Suggestion]
    {%
        Add more chocolate
    }
    
    \hint{%
        Eat in hot
    }
    
\end{recipe}

%% Complete recipe example
\begin{recipe}
  [% indications (optional)
    preparationtime = {\unit[1]{h}},
    bakingtime={\unit[1]{h}},
    bakingtemperature={\protect\bakingtemperature{
        fanoven=\unit[230]{\textcelcius},
        topbottomheat=\unit[195]{\textcelcius},
        topheat=\unit[195]{\textcelcius},
        gasstove=Level 2}},
    portion = {\portion{5-6}},
    calory={\unit[3]{kJ}},
    source = {Somebody}
]
% title
{Test Recipe}
    
    \graph
    {%
        small=pics/glass,     % small picture
        big=pics/ingredients  % big picture
    }
    
    \introduction{%
        This is an amazing recipe
    }
    
    \ingredients)\\
        3 & Eggs\\
        \unit[200]{ml} & Cream\\
        40 g & Sugar\\
        50 g & Butter
    }
    
    \preparation{%
        \step Mix all the ingredients
        \step Put them in the oven
        \step Eat it
    }
    
    \suggestion[Suggestion]
    {%
        Add more chocolate
    }
    
    \hint{%
        Eat in hot
    }
    
\end{recipe}

%% Complete recipe example
\begin{recipe}
  [% indications (optional)
    preparationtime = {\unit[1]{h}},
    bakingtime={\unit[1]{h}},
    bakingtemperature={\protect\bakingtemperature{
        fanoven=\unit[230]{\textcelcius},
        topbottomheat=\unit[195]{\textcelcius},
        topheat=\unit[195]{\textcelcius},
        gasstove=Level 2}},
    portion = {\portion{5-6}},
    calory={\unit[3]{kJ}},
    source = {Somebody}
]
% title
{Test Recipe}
    
    \graph
    {%
        small=pics/glass,     % small picture
        big=pics/ingredients  % big picture
    }
    
    \introduction{%
        This is an amazing recipe
    }
    
    \ingredients)\\
        3 & Eggs\\
        \unit[200]{ml} & Cream\\
        40 g & Sugar\\
        50 g & Butter
    }
    
    \preparation{%
        \step Mix all the ingredients
        \step Put them in the oven
        \step Eat it
    }
    
    \suggestion[Suggestion]
    {%
        Add more chocolate
    }
    
    \hint{%
        Eat in hot
    }
    
\end{recipe}

%% Complete recipe example
\begin{recipe}
  [% indications (optional)
    preparationtime = {\unit[1]{h}},
    bakingtime={\unit[1]{h}},
    bakingtemperature={\protect\bakingtemperature{
        fanoven=\unit[230]{\textcelcius},
        topbottomheat=\unit[195]{\textcelcius},
        topheat=\unit[195]{\textcelcius},
        gasstove=Level 2}},
    portion = {\portion{5-6}},
    calory={\unit[3]{kJ}},
    source = {Somebody}
]
% title
{Test Recipe}
    
    \graph
    {%
        small=pics/glass,     % small picture
        big=pics/ingredients  % big picture
    }
    
    \introduction{%
        This is an amazing recipe
    }
    
    \ingredients)\\
        3 & Eggs\\
        \unit[200]{ml} & Cream\\
        40 g & Sugar\\
        50 g & Butter
    }
    
    \preparation{%
        \step Mix all the ingredients
        \step Put them in the oven
        \step Eat it
    }
    
    \suggestion[Suggestion]
    {%
        Add more chocolate
    }
    
    \hint{%
        Eat in hot
    }
    
\end{recipe}

%% Complete recipe example
\begin{recipe}
  [% indications (optional)
    preparationtime = {\unit[1]{h}},
    bakingtime={\unit[1]{h}},
    bakingtemperature={\protect\bakingtemperature{
        fanoven=\unit[230]{\textcelcius},
        topbottomheat=\unit[195]{\textcelcius},
        topheat=\unit[195]{\textcelcius},
        gasstove=Level 2}},
    portion = {\portion{5-6}},
    calory={\unit[3]{kJ}},
    source = {Somebody}
]
% title
{Test Recipe}
    
    \graph
    {%
        small=pics/glass,     % small picture
        big=pics/ingredients  % big picture
    }
    
    \introduction{%
        This is an amazing recipe
    }
    
    \ingredients)\\
        3 & Eggs\\
        \unit[200]{ml} & Cream\\
        40 g & Sugar\\
        50 g & Butter
    }
    
    \preparation{%
        \step Mix all the ingredients
        \step Put them in the oven
        \step Eat it
    }
    
    \suggestion[Suggestion]
    {%
        Add more chocolate
    }
    
    \hint{%
        Eat in hot
    }
    
\end{recipe}

%% Complete recipe example
\begin{recipe}
  [% indications (optional)
    preparationtime = {\unit[1]{h}},
    bakingtime={\unit[1]{h}},
    bakingtemperature={\protect\bakingtemperature{
        fanoven=\unit[230]{\textcelcius},
        topbottomheat=\unit[195]{\textcelcius},
        topheat=\unit[195]{\textcelcius},
        gasstove=Level 2}},
    portion = {\portion{5-6}},
    calory={\unit[3]{kJ}},
    source = {Somebody}
]
% title
{Test Recipe}
    
    \graph
    {%
        small=pics/glass,     % small picture
        big=pics/ingredients  % big picture
    }
    
    \introduction{%
        This is an amazing recipe
    }
    
    \ingredients)\\
        3 & Eggs\\
        \unit[200]{ml} & Cream\\
        40 g & Sugar\\
        50 g & Butter
    }
    
    \preparation{%
        \step Mix all the ingredients
        \step Put them in the oven
        \step Eat it
    }
    
    \suggestion[Suggestion]
    {%
        Add more chocolate
    }
    
    \hint{%
        Eat in hot
    }
    
\end{recipe}

%% Complete recipe example
\begin{recipe}
  [% indications (optional)
    preparationtime = {\unit[1]{h}},
    bakingtime={\unit[1]{h}},
    bakingtemperature={\protect\bakingtemperature{
        fanoven=\unit[230]{\textcelcius},
        topbottomheat=\unit[195]{\textcelcius},
        topheat=\unit[195]{\textcelcius},
        gasstove=Level 2}},
    portion = {\portion{5-6}},
    calory={\unit[3]{kJ}},
    source = {Somebody}
]
% title
{Test Recipe}
    
    \graph
    {%
        small=pics/glass,     % small picture
        big=pics/ingredients  % big picture
    }
    
    \introduction{%
        This is an amazing recipe
    }
    
    \ingredients)\\
        3 & Eggs\\
        \unit[200]{ml} & Cream\\
        40 g & Sugar\\
        50 g & Butter
    }
    
    \preparation{%
        \step Mix all the ingredients
        \step Put them in the oven
        \step Eat it
    }
    
    \suggestion[Suggestion]
    {%
        Add more chocolate
    }
    
    \hint{%
        Eat in hot
    }
    
\end{recipe}


\section{Groceries}
Food to buy at the supermarket:

\begin{enumerate}
\item Rice
\item Pasta
\item Olive oil
\item Frozen vegeteables
\end{enumerate}


\section{Utensils}
How to cook

\begin{enumerate}
  \item Frying pan
  \item Boiling pot
  \item Pressure cooker
  \item Fuente barro (portuguesa)
  \item Olla hierro fundido
  \item Slow cooker
  \item Paella
  \item Wok
  \item Marinating bags
  \item Minipymer
  \item Yaogurt maker
  \item Broths farm
  \item Steamer
\end{enumerate}


\end{document} 
